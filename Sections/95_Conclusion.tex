\section{Conclusion}

The aim of this project was to analyse the performance of \acrshort{gp} based
controllers for use in longer lasting implementations, where differences in
building behaviour become important compared to the initially available data.

First, the performance of a classical \acrshort{gp} model trained on 5 days
worth of experimental data was analysed. This model turned out to be unable to
correctly extrapolate building behaviour as the weather changed throughout the
year.

Several \acrshort{svgp} implementations were then analysed. They turned out to
provide important benefits over the classical models, such as the ability to
easily scale when new data is being added and the much reduced computational
effort required. They do however present some downsides, namely increasing the
number of hyperparameters by having to choose the number of inducing locations,
as well as performing worse than then classical \acrshort{gp} implementation
given the same amount of data.

Finally, the possible improvements to the current implementations have been
addressed, noting that classical \acrshort{gp} implementations could also be
adapted to the \textit{learning control} paradigm, even if their implementation
could turn out to be much more involved and more computationally expensive than
the \acrshort{svgp} alternative.

\section*{Acknowledgements}

I would like to thank Koch Manuel Pascal for the great help provided during the
course of the project starting from the basics on CARNOT modelling, to helping
me better compare the performance of different controllers, as well as Professor
Jones, whose insights were always very guiding, while still allowing me to
discover everything on my own.

\clearpage
