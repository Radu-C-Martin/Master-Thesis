\section{Introduction}

\subsection{Motivation}

Buildings are a major consumer of energy, with more than 25\% of the total
energy consumed in the EU coming from residential
buildings~\cite{tsemekiditzeiranakiAnalysisEUResidential2019}. Combined with a
steady increase in energy demand and stricter requirements on energy
efficiency~\cite{europeancommission.jointresearchcentre.EnergyConsumptionEnergy2018},
this amplifies the need for more accessible means of regulating energy usage of
new and existing buildings.

Data-driven methods of building identification and control prove very useful
through their ease of implementation, foregoing the need of more complex
physics-based models. On the flip side, additional attention is required to the
design of these control schemes, as the results could vary greatly from one
implementation to another.

Gaussian Processes have been previously used to model building dynamics, but
they are usually limited by a fixed computational
budget~\cite{jainLearningControlUsing2018,nghiemDatadrivenDemandResponse2017}.
This limits the approaches that can be taken for identification and update of
said models.  Learning \acrfull{gp} models have also been previously used in
the context of autonomous racing cars
\cite{kabzanLearningBasedModelPredictive2019}, but there the Sparse
\acrshort{gp} model was built on top of a white-box model and only responsible
for fitting the unmodeled dynamics.

\subsection{Previous Research}
With the increase in computational power and availability of data  over time,
the accessibility of data-driven methods for system identification and control
has also risen significantly. 

The idea of using Gaussian Processes as regression models for control of dynamic
systems is not new, and has already been explored a number of times. A general
description of their use, along with the necessary theory and some example
implementations is given in~\cite{kocijanModellingControlDynamic2016}.
In~\cite{pleweSupervisoryModelPredictive2020}, a \acrshort{gp} Model with a
\acrlong{rq} Kernel is used for temperature set point optimization.

Gaussian Processes for building control have also been studied before in the
context of Demand Response~\cite{nghiemDatadrivenDemandResponse2017,
jainLearningControlUsing2018}, where the buildings are used for their heat
capacity in order to reduce the stress on energy providers during peak load
times.

There are, however, multiple limitations with these approaches. 
In~\cite{nghiemDatadrivenDemandResponse2017} the model is only identified once,
ignoring changes in weather or plant parameters, which could lead to different
dynamics. This is addressed in \cite{jainLearningControlUsing2018} by
re-identifying the model every two weeks using new information. Another
limitation is that of the scalability of the \acrshort{gp}s, which become
prohibitively expensive from a computational point of view when too much data is
added.

Outside of the context of building control, Sparse \acrlong{gp}es have been used
in autonomous racing in order to complement the physics-based model by fitting
the unmodeled dynamics of the
system~\cite{kabzanLearningBasedModelPredictive2019}.

\subsection{Contribution}

The ability to learn the plant's behaviour in new regions is very helpful in
maintaining model performance over time, as its behaviour starts deviating and
the original identified model goes further and further into the extrapolated
regions.

This project tries to combine the use of online learning control schemes with
\acrshort{gp} Models through implementing \acrfull{svgp} Models. \acrshort{svgp}s
provide means of extending the use of \acrshort{gp}s to larger datasets, thus
enabling the periodic re-training of the model to include all the historically
available data.

\subsection{Project Outline}

The main body of work can be divided in two parts: the development of a computer
model of the \pdome\ building and the synthesis, validation and comparison of
multiple control schemes using both classical \acrshort{gp}s, as well as
\acrshort{svgp}s.

Section~\ref{sec:gaussian_processes} provides the mathematical background for
understanding \acrshort{gp}s, as well as the definition in very broad strokes of
\acrshort{svgp}s and their differences from the classical implementation of
\acrshort{gp}s. This information is later used for comparing their performances
and outlining their respective pros and cons.

Section~\ref{sec:CARNOT} goes into the details of the implementation of the
\pdome\ computer model. The structure of the CARNOT model is described, and all
the parameters required for the simulation are either directly sourced from
existing literature, computed from available values or estimated using publicly
available data from \textit{Google Maps}~\cite{GoogleMaps}.

Following that, Section~\ref{sec:hyperparameters} discusses the choice of the
hyperparameters for the \acrshort{gp} and \acrshort{svgp} models respectively in
the context of their multi-step ahead simulation performance.

The \acrlong{ocp} implemented in the \acrshort{mpc} controller is described in
Section~\ref{sec:mpc_problem}, followed by a description of the complete
controller implementation in Python in Section~\ref{sec:implementation}.

Section~\ref{sec:results} presents and analyzes the results of full-year
simulations for both \acrshort{gp} and \acrshort{svgp} models. A few variations
on the initial \acrshort{svgp} model are further analyzed in order to identify
the most important parameteres for long time operation.

Finally, Section~\ref{sec:conclusion} provides a review of all the different
implementation and  discusses the possible improvements to both the current
\acrshort{gp} and \acrshort{svgp} models.

\clearpage
