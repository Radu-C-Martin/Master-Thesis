\section{Further Research}

Section~\ref{sec:results} has presented and compared the results of a full-year
simulation for a classical \acrshort{gp} model, as well as a few incarnations of
\acrshort{svgp} models. The results show that the \acrshort{svgp} have much
better performance, mainly due to the possibility of updating the model
throughout the year. The \acrshort{svgp} models also present a computational
cost advantage both in training and in evaluation, due to several approximations
shown in Section~\ref{sec:gaussian_processes}.

Focusing on the \acrlong{gp} models, there could be several ways of improving
its performance, as noted previously: a more varied identification dataset and
smart update of a fixed-size data dictionary according to information gain,
could mitigate the present problems.

Using a Sparse \acrshort{gp} without replacing the maximum log likelihood
with the \acrshort{elbo} could improve performance of the \acrshort{gp} model at
the expense of training time.

An additional change that could be made is inclusion of the most amount of prior
information possible through setting a more refined kernel, as well as adding
prior information on all the model hyperparameters when available. This approach
however goes against the "spirit" of black-box approaches, since significant
insight into the physics of the plant is required in order to properly model and
implement this information.

On the \acrshort{svgp} side, several changes could also be proposed, which were
not properly addressed in this work. First, the size of the inducing dataset was
chosen experimentally until it was found to accurately reproduce the manually
collected experimental data. In order to better use the available computational
resources, this value could be found programmatically in a way to minimize
evaluation time, while still providing good performance. Another possibility is
the periodic re-evaluation of this value when new data comes in, since as more
and more data is collected the model becomes more complex, and in general more
inducing locations could be necessary to properly reproduce the training data.

Finally, none of the presented controllers take into account occupancy rates or
adapt to possible changes in the real building, such as adding or removing
furniture, deteriorating insulation and so on. The presented update methods only
deals with adding information on behaviour in different state space regions, i.e
\textit{learning}, and their ability to \textit{adapt} to changes in the actual
plant's behaviour should be further addressed.
