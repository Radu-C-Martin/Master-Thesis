\section{Introduction}

Buildings are a major consumer of energy, with more than 25\% of the total
energy consumed in the EU coming from residential
buildings~\cite{tsemekiditzeiranakiAnalysisEUResidential2019}. Combined with a
steady increase in energy demand and stricter requirements on energy
efficiency~\cite{europeancommission.jointresearchcentre.EnergyConsumptionEnergy2018},
this amplifies the need for more accessible means of regulating energy usage of
new and existing buildings.

Data-driven methods of building identification and control prove very useful
through their ease of implementation, foregoing the need of more complex
physics-based models. On the flip side, additional attention is required to the
design of these control schemes, as the results could vary greatly from one
implementation to another.

Gaussian Processes have been previously used to model building dynamics, but
they are usually limited by a fixed computational budget. This limits the
approaches that can be taken for identification and update of said models.
Learning \acrshort{gp} models have also been previously used in the context of
autonomous racing cars, but there the Sparse \acrshort{gp} model was built on
top of a white-box model and only responsible for fitting the unmodeled
dynamics. 

This project means to provide a further expansion of the use of black-box
\acrlong{gp} Models in the context of building control, through online learning
of building dynamics at new operating points as more data gets collected.
