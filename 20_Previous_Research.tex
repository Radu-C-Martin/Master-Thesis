\section{Previous Research}
With the increase in computational power and availability of data  over time,
the accessibility of data-driven methods for System Identification and Control
has also risen significantly. 

The idea of using Gaussian Processes as regression models for control of dynamic
systems is not new, and has already been explored a number of times. A general
description of their use, along with the necessary theory and some example
implementations is given in~\cite{kocijanModellingControlDynamic2016}.
In~\cite{pleweSupervisoryModelPredictive2020} a \acrlong{gp} Model with a
\acrlong{rq} Kernel is used for temperature set point optimization.

Gaussian Processes for building control have also been studied before in the
context of Demand Response~\cite{nghiemDatadrivenDemandResponse2017,
jainLearningControlUsing2018}, where the buildings are used for their heat
capacity in order to reduce the stress on energy providers during peak load
times.

There are, however, multiple limitations with these approaches. 
In~\cite{nghiemDatadrivenDemandResponse2017} the model is only identified once,
ignoring changes in weather or plant parameters, which could lead to different
dynamics. This is addressed in \cite{jainLearningControlUsing2018} by
re-identifying the model every two weeks using new information. Another
limitation is that of the scalability of the \acrshort{gp}s, which become
prohibitively expensive from a computational point of view when too much data is
added.

Outside of the context of building control, Sparse \acrlong{gp}es have been used
in autonomous racing in order to complement the physics-based model by fitting
the unmodeled dynamics of the
system~\cite{kabzanLearningBasedModelPredictive2019}.

The ability to learn the plant's behaviour in new regions is very helpful in
maintaining model performance over time, as its behaviour starts deviating and
the original identified model goes further and further into the extrapolated
regions.


This project will therefore try to combine the use of online learning schemes
with \acrlong{gp}es by implementing \acrlong{svgp}es, which provide means of
employing \acrshort{gp} Models on larger datasets, and re-training the models
every day at midnight to include all the historically available data.

\clearpage
